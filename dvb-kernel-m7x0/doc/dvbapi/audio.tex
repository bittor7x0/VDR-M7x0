\devsec{DVB Audio Device}

The DVB audio device controls the MPEG2 audio decoder of the DVB hardware.
It can be accessed through \texttt{/dev/dvb/adapter0/audio0}.
Data types and and ioctl definitions can be accessed by including
\texttt{linux/dvb/video.h} in your application.

Please note that some DVB cards don't have their own
MPEG decoder, which results in the omission of the audio and video
device.


\devsubsec{Audio Data Types}

This section describes the structures, data types and defines used when 
talking to the audio device.

\devsubsubsec{audio\_stream\_source\_t}
\label{audiostreamsource}
The audio stream source is set through the AUDIO\_SELECT\_SOURCE
call and can take the following values, depending on whether we are
replaying from an internal (demux) or external (user write) source.
\begin{verbatim}
typedef enum {
        AUDIO_SOURCE_DEMUX, 
        AUDIO_SOURCE_MEMORY 
} audio_stream_source_t;
\end{verbatim}
AUDIO\_SOURCE\_DEMUX selects the demultiplexer (fed
either by the frontend or the DVR device) as the source of 
the video stream.
If AUDIO\_SOURCE\_MEMORY is selected the stream 
comes from the application through the \texttt{write()} 
system call.

\devsubsubsec{audio\_play\_state\_t}
The following values can be returned by the AUDIO\_GET\_STATUS call
representing the state of audio playback.
\label{audioplaystate}
\begin{verbatim}
typedef enum { 
        AUDIO_STOPPED,
        AUDIO_PLAYING,
        AUDIO_PAUSED  
} audio_play_state_t;
\end{verbatim}

\devsubsubsec{audio\_channel\_select\_t}
\label{audiochannelselect}
The audio channel selected via AUDIO\_CHANNEL\_SELECT is determined by
the following values.
\begin{verbatim}
typedef enum {
        AUDIO_STEREO,
        AUDIO_MONO_LEFT, 
        AUDIO_MONO_RIGHT, 
} audio_channel_select_t;
\end{verbatim}

\devsubsubsec{struct audio\_status}
\label{audiostatus}
The AUDIO\_GET\_STATUS call returns the following structure informing
about various states of the playback operation.
\begin{verbatim}
typedef struct audio_status { 
        boolean AV_sync_state;
        boolean mute_state;  
        audio_play_state_t play_state;
        audio_stream_source_t stream_source; 
        audio_channel_select_t channel_select;
        boolean bypass_mode;
} audio_status_t;
\end{verbatim}

\devsubsubsec{struct audio\_mixer}
\label{audiomixer}
The following structure is used by the AUDIO\_SET\_MIXER call to set
the audio volume.
\begin{verbatim}
typedef struct audio_mixer { 
        unsigned int volume_left;
        unsigned int volume_right;
} audio_mixer_t;
\end{verbatim}

\devsubsubsec{audio encodings}
\label{audiotypes}
A call to AUDIO\_GET\_CAPABILITIES returns an unsigned integer with
the following bits set according to the hardwares capabilities.
\begin{verbatim}
#define AUDIO_CAP_DTS    1
#define AUDIO_CAP_LPCM   2
#define AUDIO_CAP_MP1    4
#define AUDIO_CAP_MP2    8
#define AUDIO_CAP_MP3   16
#define AUDIO_CAP_AAC   32
#define AUDIO_CAP_OGG   64
#define AUDIO_CAP_SDDS 128
#define AUDIO_CAP_AC3  256
\end{verbatim}


\devsubsubsec{struct audio\_karaoke}
\label{audiokaraoke}
The ioctl AUDIO\_SET\_KARAOKE uses the following format:
\begin{verbatim}
typedef
struct audio_karaoke{
        int vocal1;  
        int vocal2;  
        int melody;  
} audio_karaoke_t;
\end{verbatim}

If Vocal1 or Vocal2 are non-zero, they get mixed
into left and right t at 70\% each.
If both, Vocal1 and Vocal2 are non-zero, Vocal1 gets
mixed into the left channel and
Vocal2 into the right channel at 100\% each.
Ff Melody is non-zero, the melody channel gets mixed
into left and right.

\devsubsubsec{audio attributes}
\label{aattrib}
The following attributes can be set by a call to AUDIO\_SET\_ATTRIBUTES:
\begin{verbatim}
typedef uint16_t audio_attributes_t;
/*   bits: descr. */
/*   15-13 audio coding mode (0=ac3, 2=mpeg1, 3=mpeg2ext, 4=LPCM, 6=DTS, */
/*   12    multichannel extension */
/*   11-10 audio type (0=not spec, 1=language included) */
/*    9- 8 audio application mode (0=not spec, 1=karaoke, 2=surround) */
/*    7- 6 Quantization / DRC (mpeg audio: 1=DRC exists)(lpcm: 0=16bit,  */
/*    5- 4 Sample frequency fs (0=48kHz, 1=96kHz) */
/*    2- 0 number of audio channels (n+1 channels) */
\end{verbatim}


\clearpage

\devsubsec{Audio Function Calls}

\function{open()}{
  int open(const char *deviceName, int flags);}{
  This system call opens a named audio device (e.g. /dev/dvb/adapter0/audio0) for subsequent
  use. When an open() call has succeeded, the device will be ready for use.
  The significance of blocking or non-blocking mode is described in the 
  documentation for functions where there is a difference. It does not affect 
  the semantics of the open() call itself. A device opened in blocking mode can
  later be put into non-blocking mode (and vice versa) using the F\_SETFL command
  of the fcntl system call.  This is a standard system call, documented in the 
  Linux manual page for fcntl.
  Only one user can open the Audio  Device in O\_RDWR mode. All other attempts to
  open the device in this mode will fail, and an error code will be returned.
  If the Audio Device is opened in O\_RDONLY mode, the only ioctl call that can 
  be used is AUDIO\_GET\_STATUS. All other call will return with an error code.
  }{
  const char *deviceName & Name of specific audio device.\\
  int flags & A bit-wise OR of the following flags:\\
            & \hspace{1em} O\_RDONLY read-only access\\
            & \hspace{1em} O\_RDWR read/write access\\
            & \hspace{1em} O\_NONBLOCK open in non-blocking mode \\
            & \hspace{1em} (blocking mode is the default)\\
  }{
  ENODEV    & Device driver not loaded/available.\\
  EINTERNAL & Internal error.\\
  EBUSY     & Device or resource busy.\\
  EINVAL    & Invalid argument.\\
}

\function{close()}{
  int close(int fd);}{
  This system call closes a previously opened audio device.
  }{
  int fd & File descriptor returned by a previous call to open().\\
  }{
  EBADF & fd is not a valid open file descriptor.\\
}

\function{write()}{
  size\_t write(int fd, const void *buf, size\_t count);}{
  This system call can only be used if AUDIO\_SOURCE\_MEMORY is selected 
  in the ioctl call AUDIO\_SELECT\_SOURCE. 
  The data provided shall be in PES format.
  If O\_NONBLOCK is not specified the function will block until buffer space is
  available. The amount of data to be transferred is implied by count.
  }{
  int fd      & File descriptor returned by a previous call to open().\\
  void *buf   & Pointer to the buffer containing the PES data.\\
  size\_t count& Size of buf.\\
  }{
  EPERM&  Mode AUDIO\_SOURCE\_MEMORY not selected.\\
  ENOMEM& Attempted to write more data than the internal buffer can hold.\\
  EBADF&  fd is not a valid open file descriptor.\\
}

\ifunction{AUDIO\_STOP}{
  int ioctl(int fd, int request = AUDIO\_STOP);}{
  This ioctl call asks the Audio Device to stop playing the current stream.
  }{
  int fd & File descriptor returned by a previous call to open().\\
  int request& Equals AUDIO\_STOP for this command.
  }{
  EBADF&      fd is not a valid open file descriptor \\
  EINTERNAL & Internal error.
}

\ifunction{AUDIO\_PLAY}{
  int ioctl(int fd, int request = AUDIO\_PLAY);}{
  This ioctl call asks the Audio Device to start playing an audio stream
  from the selected source.
  }{
  int fd & File descriptor returned by a previous call to open().\\
  int request& Equals AUDIO\_PLAY for this command.
  }{
  EBADF&      fd is not a valid open file descriptor \\
  EINTERNAL & Internal error.
}

\ifunction{AUDIO\_PAUSE}{
  int ioctl(int fd, int request = AUDIO\_PAUSE);}{
  This ioctl call suspends the audio stream being played. 
  Decoding and playing are paused. 
  It is then possible to restart again decoding and playing process of the
  audio stream using AUDIO\_CONTINUE command.\\
  If  AUDIO\_SOURCE\_MEMORY is selected in the ioctl call 
  AUDIO\_SELECT\_SOURCE, the DVB-subsystem will not decode (consume) 
  any more data until the ioctl  call
  AUDIO\_CONTINUE or AUDIO\_PLAY is performed.
  }{
  int fd & File descriptor returned by a previous call to open().\\
  int request& Equals AUDIO\_PAUSE for this command.
  }{
  EBADF&      fd is not a valid open file descriptor.\\
  EINTERNAL & Internal error.
}

\ifunction{AUDIO\_SELECT\_SOURCE}{
  int ioctl(int fd, int request = AUDIO\_SELECT\_SOURCE, 
  audio\_stream\_source\_t source);}{
  This ioctl call informs the audio device which source shall be used for the 
  input data. The possible sources are demux or memory. 
  If AUDIO\_SOURCE\_MEMORY 
  is selected, the data is fed to the Audio Device through the write command.
  }{
  int fd & File descriptor returned by a previous call to open().\\
  int request & Equals AUDIO\_SELECT\_SOURCE for this command.\\
  audio\_stream\_source\_t source& Indicates the source that shall be used for the
  Audio stream.
  }{
  EBADF&      fd is not a valid open file descriptor.\\
  EINTERNAL & Internal error.\\
  EINVAL    & Illegal input parameter.
}

\ifunction{AUDIO\_SET\_MUTE}{
  int ioctl(int fd, int request = AUDIO\_SET\_MUTE, boolean state);}{
  This ioctl call asks the audio device to mute the stream that is 
  currently being played.
  }{
  int fd & File descriptor returned by a previous call to open().\\
  int request & Equals AUDIO\_SET\_MUTE for this command.\\
  boolean state & Indicates if audio device shall mute or not.\\
  &TRUE     Audio Mute\\
  &FALSE   Audio Un-mute\\
  }{
  EBADF&      fd is not a valid open file descriptor.\\
  EINTERNAL & Internal error.\\
  EINVAL    & Illegal input parameter.
}

\ifunction{AUDIO\_SET\_AV\_SYNC}{
  int ioctl(int fd, int request = AUDIO\_SET\_AV\_SYNC, boolean state);}{
  This ioctl call asks the Audio Device to turn ON or OFF A/V synchronization.
  }{
  int fd & File descriptor returned by a previous call to open().\\
  int request & Equals AUDIO\_AV\_SYNC for this command.\\
  boolean state& Tells the DVB subsystem if A/V 
  synchronization shall be ON or OFF.\\
  & TRUE   AV-sync ON \\
  & FALSE  AV-sync OFF\\
  }{
  EBADF&      fd is not a valid open file descriptor.\\
  EINTERNAL & Internal error.\\
  EINVAL    & Illegal input parameter.
}

\ifunction{AUDIO\_SET\_BYPASS\_MODE}{
  int ioctl(int fd, int request = AUDIO\_SET\_BYPASS\_MODE, boolean mode);}{
  This ioctl call asks the Audio Device to bypass the Audio decoder and forward
  the stream without decoding. This mode shall be used if streams that can't be
  handled by the DVB system shall be decoded.
  Dolby DigitalTM streams are automatically forwarded by the DVB 
  subsystem if the hardware can handle it.
  }{
  int fd & File descriptor returned by a previous call to open().\\
  int request & Equals AUDIO\_SET\_BYPASS\_MODE for this command.\\
  boolean mode& Enables or disables the decoding of the current
  Audio stream in the DVB subsystem.\\
  &TRUE    Bypass is disabled\\
  &FALSE  Bypass is enabled\\
  }{
  EBADF&      fd is not a valid open file descriptor.\\
  EINTERNAL & Internal error.\\
  EINVAL    & Illegal input parameter.
}

\ifunction{AUDIO\_CHANNEL\_SELECT}{
  int ioctl(int fd, int request = AUDIO\_CHANNEL\_SELECT, 
  audio\_channel\_select\_t);}{
  This ioctl call asks the Audio Device to select the requested channel 
  if possible.
  }{
  int fd & File descriptor returned by a previous call to open().\\
  int request & Equals AUDIO\_CHANNEL\_SELECT for this command.\\
  audio\_channel\_select\_t ch &
  Select the output format of the audio (mono left/right, stereo).
  }{
  EBADF&      fd is not a valid open file descriptor.\\
  EINTERNAL & Internal error.\\
  EINVAL    & Illegal input parameter ch.
}

\ifunction{AUDIO\_GET\_STATUS}{
  int ioctl(int fd, int request = AUDIO\_GET\_STATUS, 
  struct audio\_status *status);}{
  This ioctl call asks the Audio Device to return the current state 
  of the Audio Device.
  }{
  int fd & File descriptor returned by a previous call to open().\\
  int request & Equals AUDIO\_GET\_STATUS for this command.\\
  struct audio\_status *status & Returns the current state of Audio Device.
  }{
  EBADF&      fd is not a valid open file descriptor.\\
  EINTERNAL & Internal error.\\
  EFAULT & status points to invalid address.
}

\ifunction{AUDIO\_GET\_CAPABILITIES}{
  int ioctl(int fd, int request = AUDIO\_GET\_CAPABILITIES, 
  unsigned int *cap);}{
  This ioctl call asks the Audio Device to tell us about the 
  decoding capabilities of the audio hardware.
  }{
  int fd & File descriptor returned by a previous call to open().\\
  int request & Equals AUDIO\_GET\_CAPABILITIES for this command.\\
  unsigned int *cap & Returns a bit array of supported sound formats.
  }{
  EBADF&      fd is not a valid open file descriptor.\\
  EINTERNAL & Internal error.\\
  EFAULT & cap points to an invalid address.
}

\ifunction{AUDIO\_CLEAR\_BUFFER}{
  int ioctl(int fd, int request = AUDIO\_CLEAR\_BUFFER);}{
  This ioctl call asks the Audio Device to clear all software 
  and hardware buffers of the audio decoder device.
  }{
  int fd & File descriptor returned by a previous call to open().\\
  int request & Equals AUDIO\_CLEAR\_BUFFER for this command.
  }{
  EBADF&      fd is not a valid open file descriptor.\\
  EINTERNAL & Internal error.
}

\ifunction{AUDIO\_SET\_ID}{
  int ioctl(int fd, int request = AUDIO\_SET\_ID, int id);}{
  This ioctl selects which sub-stream is to be decoded if a program or
  system stream is sent to the video device. If no audio stream type is set
  the id has to be in [0xC0,0xDF] for MPEG sound, in [0x80,0x87] for
  AC3 and in [0xA0,0xA7] for LPCM. More specifications may follow
  for other stream types. If the stream type is set the id just
  specifies the substream id of the audio stream and only the first 5
  bits are recognized.
  }{
  int fd & File descriptor returned by a previous call to open().\\
  int request & Equals AUDIO\_SET\_ID for this command.\\
  int id& audio sub-stream id
  }{
  EBADF&      fd is not a valid open file descriptor.\\
  EINTERNAL & Internal error.\\
  EINVAL & Invalid sub-stream id.
}

\ifunction{AUDIO\_SET\_MIXER}{
  int ioctl(int fd, int request = AUDIO\_SET\_MIXER, audio\_mixer\_t *mix);}{
  This ioctl lets you adjust the mixer settings of the audio decoder.
  }{
  int fd & File descriptor returned by a previous call to open().\\
  int request & Equals AUDIO\_SET\_ID for this command.\\
  audio\_mixer\_t *mix& mixer settings.
  }{
  EBADF&      fd is not a valid open file descriptor.\\
  EINTERNAL & Internal error.\\
  EFAULT & mix points to an invalid address.
}

\ifunction{AUDIO\_SET\_STREAMTYPE}{
  int ioctl(fd, int request = AUDIO\_SET\_STREAMTYPE, int type);}{
  This ioctl tells the driver which kind of audio stream to expect.
  This is useful if the stream offers several audio sub-streams 
  like LPCM and AC3.
  }{
  int fd      & File descriptor returned by a previous call to open().\\
  int request & Equals AUDIO\_SET\_STREAMTYPE for this command.\\
  int type & stream type\\
  }{
  EBADF& fd is not a valid open file descriptor \\
  EINVAL& type is not a valid or supported stream type.\\
}


\ifunction{AUDIO\_SET\_EXT\_ID}{
  int ioctl(fd, int request = AUDIO\_SET\_EXT\_ID, int id);}{
  This ioctl can be used to set the extension id for MPEG streams in 
  DVD playback. Only the first 3 bits are recognized. 
  }{
  int fd      & File descriptor returned by a previous call to open().\\
  int request & Equals AUDIO\_SET\_EXT\_ID for this command.\\
  int id & audio sub\_stream\_id\\
  }{
  EBADF& fd is not a valid open file descriptor \\
  EINVAL& id  is not a valid id.\\
}

\ifunction{AUDIO\_SET\_ATTRIBUTES}{
  int ioctl(fd, int request = AUDIO\_SET\_ATTRIBUTES, audio\_attributes\_t attr );}{
  This ioctl is intended for DVD playback and allows you to set
  certain information about the audio stream.
  }{
  int fd      & File descriptor returned by a previous call to open().\\
  int request & Equals AUDIO\_SET\_ATTRIBUTES for this command.\\
  audio\_attributes\_t attr & audio attributes according to section \ref{aattrib}\\
  }{
  EBADF& fd is not a valid open file descriptor \\
  EINVAL& attr is not a valid or supported attribute setting.\\
}

\ifunction{AUDIO\_SET\_KARAOKE}{
  int ioctl(fd, int request = AUDIO\_SET\_STREAMTYPE, audio\_karaoke\_t *karaoke);}{
  This ioctl allows one to set the mixer settings for a karaoke DVD.
  }{
  int fd      & File descriptor returned by a previous call to open().\\
  int request & Equals AUDIO\_SET\_STREAMTYPE for this command.\\
  audio\_karaoke\_t *karaoke & karaoke settings according to section \ref{audiokaraoke}.\\
  }{
  EBADF & fd is not a valid open file descriptor \\
  EINVAL& karaoke is not a valid or supported karaoke setting.\\
}

%%% Local Variables: 
%%% mode: latex
%%% TeX-master: "dvbapi"
%%% End: 
